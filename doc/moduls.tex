
\section{Mòduls del projecte}

El projecte s'ha dividit essencialment en tres mòduls, la funció dels quals es 
descriu breument a continuació. Així mateix, en seccions posteriors es 
descriuen més detalladament algunes parts del projecte; concretament, aquelles 
parts que hem considerat més importants per a la correcta comprensió del 
projecte.

\subsection{Analitzador lèxico-sintàctic}
Com a gairebé qualsevol llenguatge, a la primera part del \textit{pipeline} té 
lloc l'anàlisi lèxica i sintàctica. Aquest mòdul, generat automàticament amb 
\texttt{ANTLR}, serà el que rebi un fitxer de codi com a entrada i retorni, si 
el codi resulta ser correcte a nivell sintàctic, el seu AST corresponent. En 
aquesta secció considerarem que un codi és correcte si aquest es pot 
obtenir com a resultat de les regles de producció de la gramàtica adjunta al 
\autoref{lst:gramatica}.

\subsection{Analitzador semàntic}
Un cop generat l'arbre sintàctic, cal realitzar unes certes comprovacions que 
assegurin que el programa presentat inicialment és totalment correcte i 
traduïble. Al marge de les condicions habituals presents a la majoria de 
llenguatges, cal assegurar-se també que no existeix cap successió de crides a 
funcions que provoquin un cicle. Amb aquesta finalitat, el que es fa és
generar un graf dirigit on cada funció és un node i, donats dos nodes \(f\) 
i \(g\), només existeix un arc \((f,g)\) si \(f\), en algun punt (o en 
diversos) del seu codi, crida a \(g\). Considerarem que un programa és correcte 
respecte de les crides quan el graf generat sigui acíclic. Aquesta restricció 
imposada elimina, per tant, l'existència de qualsevol tipus de recursivitat 
(però no les crides multinivell). L'arbre sintàctic generat a la primera fase 
i el graf de crides derivat d'aquest són l'entrada que rep el mòdul de la 
següent (i última) fase.

\subsection{Generador de \textit{hardware}}
Finalment, un cop comprovada la consistència sintàctica i semàntica, es 
realitza la traducció a \textit{hardware} del programa que s'ha rebut 
inicialment com a entrada. Aquesta traducció consisteix a convertir el codi 
en \texttt{Verilog} sintetitzable que descrigui un circuit que implementi 
el programa inicial proposat. De manera similar a com fan molts llenguatges 
compilats, aquesta traducció consisteix en l'aplicació sistemàtica de regles 
predefinides que indiquen com s'ha d'implementar cadascuna de les construccions 
del llenguatge. Cal especificar que, en aquesta fase, el graf de crides 
prèviament generat per l'analitzador semàntic és especialment útil, ja que 
aquest ens indica quins mòduls de funcions s'han d'interconnectar.

