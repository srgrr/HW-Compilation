\section{Operacions sobre l'arbre sintàctic} \label{sec:graf}

En aquesta secció, s'ofereix una explicació més detallada de la obtenció de 
certa informació intermèdia que s'ha encabit en el mòdul d'anàlisi semàntica.

Cal recordar que aquesta part és la que s'executa immediatament desprès de 
l'anàlisi lèxico-sintàctica (si aquesta ha resultat reeixida). En aquesta part 
es comprova que les restriccions més típiques de llenguatges similars a 
l'emprat se satisfacin (per exemple, que no s'utilitzin variables no 
declarades, que no es redeclarin variables i altres normes similars). Donades 
les particularitats del llenguatge, també cal comprovar que cap de totes les 
possibles successions de crides que poden tenir lloc al programa desemboquin 
en recursivitat.

Aquesta part d'anàlisi semàntica també s'encarrega de realitzar uns certs 
càlculs necessaris per a la part posterior, la qual s'encarrega de traduir 
finalment el programa donat en un circuit descrit en \texttt{Verilog} 
sintetitzable. També es construeix, de forma simultània a les comprovacions 
semàntiques i els càlculs anteriorment esmentats, un graf $G$ i el seu 
transposat $G^{T}$, el qual conté com a nodes les funcions del programa. Una 
aresta $(f,g)$ és present a $G$ si $f$ crida a $g$ en algun punt del programa 
(o en diversos).

La tasca d'anàlisi semàntica s'ha dividit en tres etapes: preprocessament, 
recorregut explícit de l'AST generat al mòdul anterior i comprovació de 
funcions no utilitzades i recursivitat. Cal remarcar que, al disseny presentat,
les comprovacions semàntiques i els càlculs necessaris per a la següent fase 
es duen a terme de forma simultània, ja que aquests dos procediments no 
presenten cap mena de dependència entre ells.


\subsection{Preprocessament}

És útil obtenir certa informació prèviament al recorregut de l'AST obtingut a 
la fase anterior. Aquesta tasca consisteix a obtenir els noms i les referències
als nodes de les funcions declarades al programa. Amb aquestes dades es pot 
comprovar que cap nom de funció es declari més d'una vegada i que hi hagi una 
funció principal definida. Aquestes dades també permeten inicialitzar 
estructures de dades auxiliars, com ara els dos grafs de crides (els 
quals consistiran, després d'aquesta fase, en grafs amb tants nodes com 
funcions hi hagi i sense cap aresta). Si es troba qualsevol error, l'execució 
s'avorta en aquesta fase i no es passa a la següent.


\subsection{Recorregut de l'AST}

En aquesta etapa de l'anàlisi es recorre l'arbre sintàctic pròpiament dit. 
Donades les necessitats específiques del llenguatge, la classe que representa 
l'arbre sintàctic no és la pròpia d'\texttt{ANTLR}, sinó una extensió 
d'aquesta. Concretament, s'ha estès per permetre que cada node contingui 
informació sobre quina és la seva funció d'origen. Pels nodes que representen 
una crida a funció també es guarda un enter que indica quin número de crida 
representen dins la seva funció origen (és a dir, es tracta d'identificadors 
de les crides: si l'identificador d'una crida és 5, vol dir que aquest node 
representa la cinquena crida a funció que es pot trobar al codi de la seva 
funció origen). Quelcom similar es fa per a les assignacions.

L'estratègia de recorregut és senzilla: s'obté el node que representa la 
funció principal i es recorre tot l'arbre accessible des d'aquest. Durant el 
recorregut, es pot estar en diferents estats segons els tipus de node: funció, 
llista d'instruccions, instrucció i expressió. A cada estat es comproven 
diverses restriccions semàntiques i es construeixen diferents parts de les 
estructures de dades auxiliars anteriorment comentades.

En els nodes que representen funcions es comprova, en primer lloc, que el node 
en qüestió no hagi estat processat prèviament. Tot seguit, es comprova que els 
noms de les variables i paràmetres definits siguin consistents (tant entre 
ells com amb els noms d'altres funcions). Per consistència hem d'entendre que 
no hi hagi dos elements amb el mateix nom. Aprofitem per recordar que l'àmbit 
de visibilitat pel que fa a variables i paràmetres és únicament local i que, 
per tant, dues funcions poden tenir variables i paràmetres amb els mateixos 
noms sense cap problema. Si es troba algun error en aquesta part, s'avorta 
l'anàlisi sense passar a la següent fase. També s'aprofita per a inicialitzar 
localment unes estructures de dades que mantenen el comptatge d'assignacions 
realitzades a les variables o paràmetres d'aquesta funció, les quals seran 
necessàries per a poder assignar l'identificador d'assignacions anteriorment 
esmentat. Finalment, es comprova que l'última instrucció contingui una 
assignació la part esquerra de la qual sigui el propi nom de la funció, ja que 
això, en aquest llenguatge, es considera el valor de retorn. Donat que no 
existeixen efectes laterals, es pot considerar un error el fet de tenir una 
funció sense un valor de retorn correctament definit. Si totes les 
comprovacions han resultat correctes, es procedeix a recórrer la llista 
d'instruccions, la qual representa el codi de la funció. En l'anàlisi del cos 
de la funció, doncs, es manté un registre del nom de la funció d'origen i els 
noms visibles (arguments i variables).

Per a les llistes d'instruccions, simplement es processen tots els fills (els 
quals seran instruccions) de manera successiva, parant si algun d'aquests 
provoca un error.

Els nodes que representen instruccions es tracten per casos (un cas per tipus 
d'instrucció). Els casos possibles són: assignació, condicional (\textit{if}) i 
bucle (\textit{while}).

Pel cas de l'assignació, es comprova, en primer lloc, que l'identificador de 
la part esquerra sigui, en efecte, una variable visible dins l'àmbit de la 
funció a la qual s'està. Tot seguit, s'assigna un identificador d'assignació 
al node en qüestió i s'actualitza el comptatge d'aquestes. Finalment, si no hi 
ha hagut errors, es recorre l'expressió que correspon a la part dreta de 
l'assignació.

En els condicionals, es comença recorrent l'expressió que correspon a la 
condició. Si no hi ha hagut cap error, es recorre la llista d'instruccions del 
\textit{then}. Si no hi ha hagut cap error, es comprova si existeix un últim 
node del tipus llista d'instruccions (el qual representaria l'\textit{else}); 
en cas afirmatiu, també es recorre. Els bucles es processen de forma similar, 
obviant la part de l'\textit{else}.

L'últim cas a tenir en compte són els nodes que corresponen a les expressions. 
Aquests nodes també es tracten diferent segons de quin tipus siguin. Aquí 
podem distingir els següents casos: identificador, literal enter, crida a 
funció i operadors aritmètics.

Pel cas de l'identificador, simplement es comprova que aquest faci referència 
a algun dels noms de variables o paràmetres visibles en l'àmbit en qüestió.

Els literals s'ignoren i no es fa res amb ells. Això es degut a què, en 
aquesta part, no es necessita comprovar res dels seus valors: els problemes 
que aquests podrien donar en temes de correcció es detecten a la fase 
d'anàlisi sintàctica, la qual és anterior a aquesta.

Les crides a funcions es tracten com segueix: es comprova primer que la funció 
cridada existeix i que l'aritat proposada (el nombre d'arguments) coincideix 
amb l'aritat real d'aquesta. També es recorren les expressions que representen 
els paràmetres. Finalment, es recorre la funció que s'està cridant. 
Addicionalment, s'afegeixen les arestes corresponents als dos grafs de crides.

Finalment, pels operadors aritmètics o lògics es comproven recursivament les 
expressions que es tenen com a operands.


\subsection{Comprovació de recursivitat}

Un cop acabat el processament descrit a la secció anterior, es procedeix a 
comprovar que no existeixi cap successió de crides que desemboqui en 
recursivitat, ja sigui directa o indirecta. Per això, es mira que cap funció 
no es cridi a si mateixa i que el graf dirigit de crides $G$ generat no 
contingui cap component fortament connexa de més d'un vèrtex. Donat que 
l'algorisme fet servir per aquesta detecció és constructiu, es pot donar com a 
missatge d'error el conjunt (o conjunts) exacte de funcions que creen 
recursivitat. Com a efecte lateral de l'execució d'aquest algorisme, també 
s'obté una col·lecció de noms de funcions, els quals representen les funcions 
que són potencialment utilitzades en l'execució del programa. Això és 
especialment útil per a la següent fase, ja que permet fer una traducció 
selectiva, podent resultar en programes més petits que els que s'obtindrien si 
la traducció fos del tot sistemàtica. Aquesta col·lecció també es fa servir per 
emetre avisos de codi no emprat al programa final.


