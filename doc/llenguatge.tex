
\section{Especificació del llenguatge}

El llenguatge que dissenyem és molt simple i limitat. Això ens 
permet concentrar-nos en la part més interessant del projecte, que és la de 
traducció de \textit{software} a \textit{hardware}. De totes maneres, vegem 
a continuació les principals característiques del llenguatge.

Al principi de cada programa, el programador ha de fixar la mida dels vectors 
de bits que s'utilitzaran. Llavors, la sintaxi del llenguatge permet aplicar 
operacions molt senzilles sobre aquests vectors de bits: sumes i restes 
(interpretant els vectors com a enters codificats en complement a dos); 
conjuncions, disjuncions i negacions bit a bit; conjuncions, disjuncions i 
negacions lògiques (entenent que tot vector diferent de zero codifica el valor 
cert), i comparacions d'igualtat i desigualtat. La precedència dels operadors 
és la mateixa que a \texttt{C}.

Un programa està format per una o més funcions, una de les quals s'ha 
d'anomenar \texttt{main}: aquesta és, com el seu nom indica, el punt 
d'entrada a l'execució. Cadascuna d'aquestes funcions, excepte \texttt{main}, 
pot rebre alguns arguments i ha de retornar exactament un vector de bits com 
a sortida. A més, les variables utilitzades per una funció han d'haver estat 
declarades al principi d'aquesta (això facilita el procés de traducció, ja 
que, en traduir el programa a un circuit, cada variable correspon a un 
registre), i una funció només té accés a aquestes variables, així com als seus 
paràmetres. Una variable especial amb el mateix nom que la funció 
emmagatzema el valor de retorn: aquesta s'ha d'assignar a l'última línia del 
cos de la funció (per tal de facilitar l'anàlisi semàntica dels programes) i, 
a diferència de les altres variables i paràmetres, no es pot utilitzar en 
expressions dins de la pròpia funció. 

Les instruccions bàsiques del llenguatge són assignacions i sentències de 
control de flux. En particular, el llenguatge disposa d'una estructura 
condicional (\textit{if-then-else}) i una estructura repetitiva 
(\textit{while}). 

A partir de les operacions mencionades prèviament i l'ús de literals 
(especificats en bases binària, decimal o hexadecimal), variables i crides a 
funcions, es construeixen les expressions bàsiques del llenguatge.

Com ja s'ha explicat, aquest llenguatge no simula una memòria i tampoc permet 
recursivitat. Així, totes les variables a utilitzar han d'haver estat 
prèviament definides pel programador estàticament. Per tant, l'expressivitat 
del llenguatge és força limitada (tot i que és més que suficient per a 
l'objectiu d'aquest projecte).


