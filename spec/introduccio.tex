
\section{Introducció}

Tradicionalment, aprenem de forma independent i aïllada el funcionament del 
\textit{hardware} dels computadors i la programació mitjançant 
\textit{software}. Malgrat les capes d'abstracció que afegim entremig, però, 
és clar que és tècnicament possible, per a cada programa que puguem expressar 
mitjançant qualsevol llenguatge de programació, dissenyar un circuit lògic que 
l'executi.

Aquest projecte té com a objectiu fonamental mostrar que, a més, es pot fer 
aquest pas d'un programa expressat en un llenguatge de programació 
convencional a un circuit digital equivalent de forma automàtica. 

A tal efecte, dissenyem un llenguatge de programació molt simple de capacitats 
reduïdes però amb algunes de les estructures de seqüenciació més habituals i 
construirem un traductor que, a partir dels programes especificats en aquest 
llenguatge, generi automàticament els circuits corresponents. En particular, 
aquest llenguatge de programació tractarà amb vectors de bits d'una mida 
fixada per a cada programa (que, llavors, el programador pot interpretar com a 
nombres enters, per exemple) i disposarà de sentències condicionals i 
repetitives, així com d'un sistema de funcions i crides a aquestes. Una de 
les principals limitacions d'aquest llenguatge, però, serà la falta de memòria 
(cosa que, d'altra banda, facilitarà la simulació dels programes en circuits). 

Tanmateix, l'interès d'aquest projecte no deixa de ser merament acadèmic, ja 
que els circuits generats no són en general òptims (ni pretenen ser-ho) i, per 
tant, amb programes d'una complexitat mitjana ja és possible tenir una 
explosió de portes lògiques que en faci infactible la implementació a la 
pràctica. Per aquest motiu, un llenguatge limitat com el descrit és suficient 
per exemplificar la traducció automàtica de programes a circuits.

