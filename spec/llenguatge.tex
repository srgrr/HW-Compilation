
\section{Especificació del llenguatge}

El llenguatge que dissenyem serà, en principi, molt simple i limitat. Això ens 
permetrà concentra-nos en la part més interessant del projecte, que és la de 
traducció de \textit{software} a \textit{hardware}. De totes maneres, vegem 
a continuació les principals característiques del llenguatge.

Al principi de cada programa, el programador ha de fixar la mida dels vectors 
de bits que s'utilitzaran. Llavors, la sintaxi del llenguatge permet aplicar 
operacions molt senzilles sobre aquests vectors de bits: sumes i restes 
(interpretant els vectors com a enters codificats en complement a dos); 
conjuncions, disjuncions i negacions bit a bit; conjuncions, disjuncions i 
negacions lògiques (entenent que tot vector diferent de zero codifica el valor 
cert), i comparacions d'igualtat i desigualtat. La precedència dels operadors 
és la mateixa que a \texttt{C}.

Un programa està format per una o més funcions, una de les quals s'ha 
d'anomenar \texttt{main}: aquesta serà, com el seu nom indica, el punt 
d'entrada a l'execució. Cadascuna d'aquestes funcions pot rebre alguns 
arguments i ha de retornar exactament un vector de bits com a sortida. A més, 
les variables utilitzades per una funció han d'haver estat declarades al 
principi d'aquesta (això facilitarà el procés de traducció, ja que, en traduir 
el programa a un circuit, cada variable correspondrà a un registre), i una 
funció només té accés a aquestes variables, així com als seus paràmetres. Una 
variable especial amb el mateix nom que la funció emmagatzemarà el valor de 
retorn: aquesta s'haurà d'assignar a l'última línia del cos de la funció (per 
tal de facilitar l'anàlisi semàntica dels programes). 

Les instruccions bàsiques del llenguatge seran assignacions i sentències de 
control de flux. En particular, el llenguatge disposa d'una estructura 
condicional (\textit{if-then-else}) i una estructura repetitiva 
(\textit{while}). 

A partir de les operacions mencionades prèviament i l'ús de literals 
(especificats en bases binària, decimal o hexadecimal), variables i crides a 
funcions, es construeixen les expressions bàsiques del llenguatge.

Com ja s'ha explicat, aquest llenguatge no simula una memòria i tampoc permet 
recursivitat. Així, totes les variables a utilitzar han d'haver estat 
prèviament definides pel programador estàticament. Per tant, l'expressivitat 
del llenguatge és força limitada (tot i que és més que suficient per a 
l'objectiu d'aquest projecte).

Finalment, segueixen alguns exemples d'ús del llenguatge. En particular, al 
\autoref{lst:ex-enters} es mostren alguns algorismes elementals utilitzats en 
teoria de nombres, mentre que al \autoref{lst:ex-xifrat} es mostra una 
estratègia molt simple per xifrar i desxifrar blocs de dades. En aquests 
exemples es poden observar les construccions bàsiques que permet el llenguatge.

\lstinputlisting%
  [caption={Exemple d'ús per funcions en enters.},style=hw-comp,label={lst:ex-enters}]%
  {../examples/number_theory.hwc}

\lstinputlisting%
  [caption={Exemple d'ús en un mètode de xifrat.},style=hw-comp,label={lst:ex-xifrat}]%
  {../examples/tea.hwc}


